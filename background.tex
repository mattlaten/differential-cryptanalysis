

%%%%%%%%%%%%%%%%%%%%%%%%%%%%%%%%%%%%%%%%%%%%%%%%%%%%%%%%%%%%%%%%%%%
%                                                                 %
%                           Background                            %
%                                                                 %
%%%%%%%%%%%%%%%%%%%%%%%%%%%%%%%%%%%%%%%%%%%%%%%%%%%%%%%%%%%%%%%%%%%


\chapter{Background} \label{c:background}

We will start by looking at some basic Probability Theory, and then move on to
Block Ciphers and interesting properties surrounding them.  Since the amount of
Boolean Algebra required to understand this paper is minimal, a short comment
on this topic will be included in the Block Ciphers section. We will only
define concepts that are needed as stepping stones to explaining Differential
Cryptanalysis of Block Ciphers, and thus exclude some fundamental theorems to
certain sections that are not needed. \cite{Alko}

\section{Probability Theory}

What does it mean for an event having a probability of occurring? You probably
have some intuitive understanding of what this means. For example, you will
probably be aware that when you flip a regular coin, you have a 50\% chance
that it will land with heads facing up, and a 50\% chance that it will land
with tails facing up. It is also easy to see that if you roll an unweighted
die, you have a 1 in 6 chance of landing on a particular number that was chosen
before hand.

What if I ask you what the probability of you getting an even number on a die
is after you roll it. Most people would say there is a 50\% chance, since half
of the numbers are even and half of the numbers are odd. With this very
intuitive understanding of probability, we will define probability more
rigorously below.

Firstly, a \textbf{random experiment} is a procedure where the outcome cannot
be determined before the procedure is completed. In our examples above, tossing
a coin or rolling a die can be considered random experiments.  The set of all
possible outcomes to a random experiment is called the \textbf{sample space}
and a particular instance of conducting the random experiment is known as a
\textbf{trial} \cite{IntroStat} . An \textbf{event} in this context is a
subset of the sample space. So the coin landing with heads facing upwards,
or the die landing on a 3, or even the die landing on an even number would all
be examples of events occurring. However, an event that is a singleton in terms
of being a subset of the sample space is called an \textbf{elementary event}.
Thus, only getting heads on a coin toss, or getting a 3 on a die roll can be
considered elementary events. Finally we will end of with a definition about
how events relate to each other.

\begin{defn}
For $S$, some sample space, let $A, B \subset S$ be events. Then they are called \textbf{mutually exclusive}
if $A \cap B = \emptyset$.
\end{defn}

So what is probability then? Kolmogorov, often considered the Father of
probability, defined it as follows \cite{IntroStat}:

\begin{defn}
Suppose $S$ is a sample space for a random experiment. Then, for all events
$A \subset S$, we define the \textbf{probability} of $A$, denoted $Pr(A)$, to be a real
number with the following properties:
\begin{enumerate}
\item $0 \leq Pr(A) \leq 1$
\item $Pr(S) = 1$ and $Pr(\emptyset) = 0$, where $\emptyset$ is the empty set or \textbf{null event}.
\item For $A, B \subset S$, if $A \cap B = \emptyset$ then $Pr(A \cup B) = Pr(A) + Pr(B)$
\end{enumerate}
\end{defn}

Now that we have made precise the definition of probability, we can look
into calculating probabilities of events occurring. 

\begin{thm}
If $A_1, A_2, ... ,A_n$ are pairwise mutually
exclusive, or rather for $i \neq j, A_i \cap A_j = \emptyset$, then
\begin{equation}
    Pr(A_1 \cup A_2 \cup ... \cup A_n) = \
    Pr(A_1) + Pr(A_2) + ... + Pr(A_n)
\end{equation}
This can be written concisely as
\begin{equation}
    Pr\left(\bigcup _{i=1}^nA_i\right) = \sum _{i=1}^nA_i 
\end{equation}
\end{thm}

\begin{prf}
This proof can be obtained by the repeated use of Axiom 3.

We know that for any $A_i, A_j \in \{A_1, A_2, ... , A_n\}$,
$A_i$ and $A_j$ are mutually exclusive.

Thus, by use of Axiom 3, we have 
\begin{equation}
Pr\left(\bigcup _{i=1} ^n A_i\right) = Pr\left(\bigcup _{i=1} ^{n-1} A_i\right) + Pr(A_n)
\end{equation}

But it can also be noted that 

\begin{equation}
Pr\left(\bigcup _{i=1} ^{n-1} A_i\right) = Pr\left(\bigcup _{i=1} ^{n-2} A_i\right) + Pr(A_{n-1})
\end{equation}

In general, for any $3 \leq k \leq n$

\begin{equation}
Pr\left(\bigcup _{i=1} ^{k} A_i\right) = Pr\left(\bigcup _{i=1} ^{k-1} A_i\right) + Pr(A_{n-1})
\end{equation}
\end{prf}

Thus

Lead up into calculating probabilities from first principles, namely number
of elementary events in A over number of elementary events in S.

\begin{defn}
Events are called \textbf{independent} if the outcome of one event does
not affect the outcome of another. (Needs proper def for later)
\end{defn}
\begin{comment}
and event to have a probability of occurring, how we calculate 
an event's probability of occurring, and how likely an
event is, or string of random events are to occur.
\end{comment}

How to multiply independent probabilities together etc.

\section{Boolean Algebra}
In order to understand how many Block Ciphers work, we will have to take a 
look at Boolean Algebra, that is, the logical calculus of truth values.
In particular we will look at the operation known as the `exclusive or', 
commonly known as XOR and represented by the symbol `$\oplus$'.

\begin{defn}
The XOR of two boolean values is true if either one of the values is true,
and is false if both are true, or both are false.
\end{defn}
In simpler terms, we can view it as a function that takes two inputs, and 
returns true if either the one value or the other is true, but not both.
As we are dealing with True and False values, we will use the more compact
notation of representing $True$ as $1$, and $False$ as $0$. 

Thus, the truth table for the XOR operation is given as follows:
\begin{center}
\begin{tabular}{|r|r|r|}
\hline
$A$ & $B$ & $A \oplus B$ \\\hline
0 & 0 & 0 \\\hline
0 & 1 & 1 \\\hline
1 & 0 & 1 \\\hline
1 & 1 & 0 \\\hline
\end{tabular}
\end{center}


This can be compactly noted in the mutliplication table below:
\begin{center}
\begin{tabular}{|r|r|r|}
\hline
& 0 & 1 \\\hline
0 & 0 & 1 \\\hline
1 & 1 & 0 \\\hline
\end{tabular}
\end{center}

What this above table is saying, is that $0 \oplus 0 = 1 \oplus 1 = 0$.
Likewise, $0 \oplus 1 = 1 \oplus 0 = 1$.

\begin{rem}
It is easy to see that this operation is equivalent to addition in $\F_2$,
that is, the finite field of order 2.
\end{rem}

\section{Bits}

Firstly, recapping some terminology:
\begin{itemize}
\item A \textbf{bit} or binary digit is varible which holds either a $0$ or a $1$
\item A \textbf{bit string} is a sequence of 1 or more bits.
\end{itemize}

We can see that the definition of a bit fits well with our representation of
boolean values in the previous section. Thus, our definition of XOR applies
to bits as well. Furthermore, a bit string can be bit-wise XOR'd with a string
of the same length. 

\begin{example}
Suppose we wanted to XOR 101011 with 011010. We would move through both bit
strings, bit by bit, and XOR the individual bits together.
Thus, $101011 \oplus 011010 = (1 \oplus 0) + (0 \oplus 1) + (1 \oplus 1) + (0 \oplus 0) + (1 \oplus 1) + (1 \oplus 0)
                            = 110001$
where $+$ represents the concatenation of strings.
\end{example}

You might notice that this operation can be described as taking the first bit
string, and flipping the bit whenever you see a 1 as the corresponding bit in
the second bit string.

\section{Block Ciphers}
We will be considering Attacks on Block Ciphers later, and thus it makes sense
to introduce Block Ciphers as part of the background.  In short, Block Ciphers
can be defined as algorithms that operate on a fixed amount of bits, using some
sort of symmetric key. [citation needed] 

Alright, let's take a step back and try understand what that means. 

There are different types of block ciphers, but for the purposes of this paper,
we will hone in on Substitution-Permutation Networks (SPN). Other block ciphers
include Iterated block ciphers, and Feistel ciphers, which are beyond the scope
of our discussion.

Thus, in a typical block cipher, our plaintext is broken up to fixed-length
groups of bits, called blocks. What makes SPNs different from other block
cipher implementations, is the way the symmetric key is mixed in with the
plaintext to form ciphertext. 

In particular, a block cipher is a combination of 2 paired algorithms, E for
encryption, and D for decryption. Both algorithms accept 2 inputs, an input box
of size n bits, and a key of size k bits and both yield a n-bit output block. 

How could we make this definition more precise though? With our intutive
understanding of a Block Cipher, we can define it mathematically as follows:


\begin{defn}
For any $K$, an input key of bit length $k$, and $P$ is a string of input bits
Let us consider a function 
\begin{equation}
E_K(P) := E(K,P): \{0,1\}^k \times \{0,1\}^n \rightarrow \{0,1\}^n
\end{equation}
of length $n$. We label the output of this function, a string of $n$ bits, C. 
For each $K$, the function $E_K(P)$ is required to be an invertible mapping
on $\{0,1\}^n$, with the inverse defined as:
\begin{equation}
E_K ^{-1}(C) := D(K,C): \{0,1\}^k \times \{0,1\}^n \rightarrow \{0,1\}^n
\end{equation}
such that
\begin{equation}
\forall K\ D_K(E_K(P)) = P
\end{equation}
holds.

Then the pair $(E_K, D_K = E_K ^{-1})$ constitutes a block cipher. 
\end{defn}

\begin{rem}
In the above:
\begin{itemize}
\item $k$ is known as the \textbf{key size}
\item $n$ is known as the \textbf{block size}
\item $C$ is known as the \textbf{ciphertext}
\item $P$ is known as the \textbf{plaintext}
 
\end{itemize}
\end{rem}




\subsection{Vectorial Boolean Functions (S-boxes)}
A large part of understanding Block Ciphers, and how to attack them, will be
tied up in understanding S-boxes, (which are types of Vectorial Boolean
Functions). This section will deal with introducing and explaining them, along
with a few examples.


\begin{example}
Let's consider a simple 3 bit S-box. Since there are only 3 bits, we have 
$2^3 = 8$ possible inputs. Since S-boxes are bijections, we can only have
8 possible outputs. Thus, we can think of an S-box as a type of look-up
table.

\begin{center}
\begin{tabular}{|r|r|}
\hline
$x$ & $y = S(x)$ \\\hline
$000$ & $010$ \\\hline
$001$ & $110$ \\\hline
$010$ & $000$ \\\hline
$011$ & $100$ \\\hline
$100$ & $011$ \\\hline
$101$ & $001$ \\\hline
$110$ & $111$ \\\hline
$111$ & $101$ \\\hline
\end{tabular}
\end{center}

We can however reverse the S-box, taking $x$ to be the subject of our formula.
So instead of $y = S(x)$, we get $x = S^{-1}(y)$. Looking at the previous
table like this, we get:

\begin{center}
\begin{tabular}{|r|r|}
\hline
$y$ & $x = S^{-1}(y)$ \\\hline
$000$ & $010$ \\\hline
$001$ & $101$ \\\hline
$010$ & $000$ \\\hline
$011$ & $100$ \\\hline
$100$ & $011$ \\\hline
$101$ & $111$ \\\hline
$110$ & $000$ \\\hline
$111$ & $110$ \\\hline
\end{tabular}
\end{center}

\end{example}

\subsection{P-boxes}
S-boxes provide good confusion. What about diffusion? P-boxes.
Refer to S-box followed immediately by P-box as SP-box.

\section{Differential Properties of SP-boxes}
Next will be discussed the various differential properties of S-boxes which
allow cryptanalysts to mount an attack against a block cipher.  These include
the property that XORs do not affect differentials, as well as ways to find
relationships between input differentials and output differentials of an S-box.



\subsection{Probabilities of Differential Trails}
The previous properties discussed can be combined to give us insight into the
probabilities of differentials trails, or in plain English: Given an input
differential, how likely or probable is it that a certain output differential
occurs. This section will discuss the theory behind probabilities of
differentials, as well as practical examples of cases where probabilities do
not conform to the norm. 

All of this however, is based on the assumption that these probabilities are
linearly independent. If they are not, then we can't be sure of what the
final probability is.

\begin{comment}


\begin{thm} \label{t:cyclic} % \cite{Rose}  Rose pg 91

Let $r \in \N$.  If $p$ is an odd prime then $\Z_{p^r}^*$ is a cyclic group of order
$p^{r-1} \, (p-1)$.  For $r \ge 2$, $\Z_{2^r}^*$ is the direct product of two cyclic
groups of order $2$ and $2^{r-2}$ respectively.
% $\Z_{2^r}^* \cong C_2 \times C_{2^{r-2}$.  In other words, there exist elements $h$ and
% $g$ of order $2$ and $2^{r-2}$ respectively in $\Z_{2^r}^*$ such that every element $x$
% of $\Z_{2^r}^*$ can be written uniquely as $x \equiv h^s \, g^t \bmod 2^r$ for some
% $s \in \{0,1\}$ and $t \in \{0,1, \ldots, 2^{r-2}-1 \}$.  (i.e., $\Z_{2^r}^*$ has a basis
% of two elements, a generator pair.) Specifically, $h = -1$ will work.

\end{thm}


\bigskip

Theorem \ref{t:cyclic} has several immediate consequences:

% Obviously if $g$ is a generator of $\Z_{p^r}^*$ then it's also a generator of $\Z_{p^{\ell}}^*$
% for any $1 \le \ell \le r$.

\begin{lem}

Let $p$ be an odd prime, $r \in \N$ and $g$ a generator of $\Z_{p^r}^*$.  Then the
element $g^s$ has order
\[ \dfrac{p^{r-1} \, (p-1)} {\gcd \left(p^{r-1} \, (p-1), \, s \right)} \]
in $\Z_{p^r}^*$, and $g^s$ is a quadratic residue modulo $p^r$ if and only if $s$ is
even.

\end{lem}

% So exactly half the elements of $\Z_{p^r}^*$ are quadratic residues.
% This follows from the fact that $g$ is a quadratic non-residue (because
% if $g = w^2$ then $g^{\frac{p^{r-1} \, (p-1)}{2}}
% \equiv w^{p^{r-1} \, (p-1)} \equiv 1 \bmod p^r$,
% and $g$ cannot be a generator).

\begin{lem} \label{l:same generator}

Let $p$ be an odd prime, $r \in \N$ and $g$ a generator of $\Z_{p^r}^*$.  Then $g$ is
also a generator for $\Z_{p^k}^*$ for all $k < r$.

Let $(-1,h)$ be a generating pair for $\Z_{2^r}^*$.  Then it is
also a generating pair for
$\Z_{2^k}^*$ for all $k < r$. % $h$ has order $2^{r-2}$.

%This is really the same as saying if $h$ has order $2^{r-2}$ modulo $2^r$ then it has order $2^{k-2}$ modulo $2^k$
%for $k \le r$.

\end{lem}

%Note $-1 \in \Z_{2^r} \leftrightarrow (-1,1) \in C_2 \times C_{2^{r-2}}$, not $(1,-1)$.
%$-1$ is not a power of $g$.

% $g$ is a generator of $\Z_{2^r}^* / \{-1,1\}$?

\begin{thm} \label{t:square roots of 1}

Let $r \in \N$.  If $p$ is an odd prime, then
\[ x^2 \equiv 1 \bmod p^r \; \Leftrightarrow \; x \equiv \pm 1 \bmod p^r. \]
For $p = 2$ we have
\begin{align*}
x^2 \equiv 1 \bmod 2^r \;
    &\Leftrightarrow \;
        x \equiv \pm 1 \bmod 2^r \text{~~ or ~~} x \equiv \pm 1 + 2^{r-1} \bmod 2^r \\
    &\Leftrightarrow \; x \equiv \pm 1 \bmod 2^{r-1}.
\end{align*}

\end{thm}

% \begin{prf}

% If $p$ is odd let $g$ be a generator of $\Z_{p^r}^*$.  Then
% $g^{2s} \equiv 1 \bmod p^r$ iff $p^{r-1} \, (p-1) \divides 2s$
% iff $\frac{1}{2} p^{r-1} \, (p-1) \divides s$ iff $g^s \equiv \pm 1 \bmod p^r$.

% If $p = 2$ then $x^2 \equiv 1 \bmod p^r \not \Rightarrow x \equiv \pm 1 \bmod p^r$.
% To see this, note that
% \[ x^2 \equiv 1 \bmod 2^r \Leftrightarrow 2^r \divides x^2-1 = (x-1)(x+1), \]
% and $2$ can divide both $(x-1)$ and $(x+1)$. % which is the difference from other primes $p$.
% But $4$ cannot divide both though, so if
% \[ 2^j \divides (x-1) \text{ and } 2^{r-j} \divides (x+1) \]
% then we must have $j = 0,1, r-1$ or $r$.
% So $x \equiv \pm 1 \bmod 2^{r-1}$, i.e., $x = \pm 1 + c 2^{r-1}$ for some integer $c$.
% So modulo $2^r$, $x$ is $\pm 1 + (c \bmod 2) 2^{r-1}$.
% In other words, $x \equiv \pm 1 \bmod 2^r$ or $x \equiv \pm 1 + 2^{r-1} \bmod 2^r$.
% So $1$ has four square roots modulo $2$.

% $-1$ is a QNR modulo $4$, so $y^{2^k}+1$ divisible by $2$ but not by $4$.
% \qed \end{prf}

So $1$ has two square roots in $\Z_{p^r}^*$ if $p$ is an odd
prime, and four square roots in $\Z_{2^r}^*$ for each $r \ge 3$
(namely, $\pm 1$ and $\pm 1 + 2^{r-1}$).
% (For $r = 2$ $\pm 1 \equiv \mp 3 \bmod 4$.)
% i.e., $1$, $h$, $g^{2^{r-3}}$ and $h \, g^{2^{r-3}}$.

% For example, if $x = 7$ and $r = 4$, then the four square roots
% of $1$ modulo $2^4$ are $\pm 1$ and $\pm 1 + 2^3$, i.e., $9$ and $7$.
\end{comment}
